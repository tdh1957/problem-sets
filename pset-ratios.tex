\documentclass[12pt]{article}
\usepackage{amssymb,amsmath,latexsym,gensymb,mdwlist,tikz,multicol,enumitem}
\usepackage{pgf}

% Page length commands go here in the preamble
\setlength{\oddsidemargin}{0in} % Left margin of 1 in + 0 in = 1 in
\setlength{\textwidth}{6.5in}   % Right margin of 8.5 in - 1 in - 6.5 in = 1 in
\setlength{\topmargin}{-0.5in}  % Top margin of 2 in -1 in = 1 in
\setlength{\textheight}{9in}  % Lower margin of 11 in - 9 in - 1 in = 1 in

\setlength{\parindent}{0in} %Set the length of the paragraph indentation.

%%%%%%%%%%%%%%%%%%%%%%%%%%%%%%%%%%%%%%%%%%

\begin{document}
\pagestyle{empty} % controls page numbering
\begin{center}
          Problem set - Ratios \\[0.5in]
\end{center}
Name: \rule{4in}{0.005in} Date: \rule{1.5in}{0.005in} 
  \vspace{0.25in}

Solve each problem. Show all of your work. You may use a calculator and notes. 

\begin{enumerate}
		\newcommand{\spacing}{\vspace{0.60in}}
\item Write as a fraction in lowest terms the ratio 60:25.
\spacing

\item A mixture contains seven pounds of peanuts and three pounds of cashews. 
    \begin{enumerate}
	\item What is the ratio of peanuts to cashews?
	\vspace{0.125in}
	\item What is the ratio of cashews to peanuts?
	\vspace{0.125in}
	\item What is the ratio of peanuts to mixture?
	\vspace{0.125in}
	\item What is the ratio of cashews to mixture?
	\vspace{0.125in}
    \end{enumerate}
\spacing

\item In a certain mixture, the ratio of peanuts to cashews is 2:1. 
    \begin{enumerate}
	\item If the mixture contains six pounds of peanuts, how many pounds of cashews does it contain?
	\vspace{0.125in}
	\item If the mixture contains two pounds of cashews, how many pounds of peanuts does it contain?
	\vspace{0.125in}
	\item How many pounds of cashews are there in twelve pounds of mixture?
	\vspace{0.125in}
	\item How many pounds of peanuts are there in twelve pounds of mixture?
	\vspace{0.125in}
    \end{enumerate}
\spacing

\item The south store sold \$750,000 worth of product, while the north store sold \$520,000. In lowest terms, what was the ratio of sales in the south store to sales in the north store?
\spacing

\item \$81,000 is to be allocated to two accounts in a ratio of 7:2. How much goes into each account? 
\spacing 

\item In a certain mixture, the ratio of peanuts to cashews to raisins is 3:2:1. How many pounds of each ingredient is there in eighteen pounds of mixture?
\spacing

\item A grocer mixes eight pounds of peanuts, five pounds of cashews, and two pounds of raisins. In lowest terms, what is the ratio of peanuts to cashews to raisins?
\spacing

\item Process mixture A requires a ratio of carrier to active ingredient of 50:1. To the nearest liter, how many liters of each are required for 3,000 liters of mixture? 
\spacing

\item A twelve-pound bag of calcium chloride is mixed with eighteen pounds of sodium chloride and two pounds of organic surfactant. In lowest terms, what will be the ratio of the components in the mix?
\spacing

\item Eight tenths of a mixture is carrier, one tenth is surfactant, and the rest is active ingredient. What is the ratio of the components in lowest terms?
\spacing

\item Twenty-four thousand gallons of water is to be distributed to cells A, B, and C in a ratio of 4:3:1. How much does each cell get?
\spacing

\end{enumerate}
\end{document}
