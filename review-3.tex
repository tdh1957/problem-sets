\documentclass[12pt]{article}
\usepackage{amssymb,amsmath,latexsym,gensymb,mdwlist,tikz,multicol,enumitem}

% Page length commands go here in the preamble
\setlength{\oddsidemargin}{0in} % Left margin of 1 in + 0 in = 1 in
\setlength{\textwidth}{6.5in}   % Right margin of 8.5 in - 1 in - 6.5 in = 1 in
\setlength{\topmargin}{-0.5in}  % Top margin of 2 in -1 in = 1 in
\setlength{\textheight}{9in}  % Lower margin of 11 in - 9 in - 1 in = 1 in

\setlength{\parindent}{0in} %Set the length of the paragraph indentation.

%%%%%%%%%%%%%%%%%%%%%%%%%%%%%%%%%%%%%%%%%%

\begin{document}
\pagestyle{empty} % controls page numbering
\begin{center}
          Review 3  \\[0.5in]
\end{center}
Name: \rule{4in}{0.005in} Date: \rule{1.5in}{0.005in} 
  \vspace{0.25in}
Solve each problem. Show all of your work. You may use a calculator, textbook, and notes.

\begin{enumerate}
\item Put the correct sign, $=$, $<$, or $>$, between each pair of numbers. 

\begin{multicols}{2}
\begin{enumerate}

  \item \hspace{0.25in} $\dfrac{3}{8} \hspace{0.25in} \dfrac{5}{8}$
  \vspace{0.25in}

  \item \hspace{0.25in} $-\dfrac{2}{7} \hspace{0.25in} -\dfrac{1}{7}$
  \vspace{0.25in}

  \item \hspace{0.25in} $\dfrac{1}{8} \hspace{0.25in} \dfrac{1}{10}$
  \vspace{0.25in}

  \item \hspace{0.25in} $\dfrac{3}{8} \hspace{0.25in} \dfrac{6}{16}$
  \vspace{0.25in}

  \item \hspace{0.25in} $-\dfrac{3}{8} \hspace{0.25in} 0.375$
  \vspace{0.25in}

  \item \hspace{0.25in} $2.5 \hspace{0.25in} 2\dfrac{3}{10}$
  \vspace{0.25in}

  \item \hspace{0.25in} $-0.66 \hspace{0.25in} -0.72$
  \vspace{0.25in}

  \item \hspace{0.25in} $-3\dfrac{5}{8} \hspace{0.25in} -3.62$
  \vspace{0.25in}

  \item \hspace{0.25in} $0.44 \hspace{0.25in} \dfrac{4}{9}$
  \vspace{0.25in}

  \item \hspace{0.25in} $\dfrac{21}{100} \hspace{0.25in} \dfrac{208}{1000}$
  \vspace{0.25in}

\end{enumerate}
\end{multicols}

  \vspace{0.5in}

\item Write the numbers in order from least to greatest.
\begin{multicols}{2}
\begin{enumerate}

\item \hspace{0.5in} $\dfrac{1}{2}$, $0.333$, $\dfrac{5}{16}$, $0.125$ 
  \vspace{0.5in}

\item \hspace{0.5in} $1\dfrac{5}{8}$, $-0.500$, $\dfrac{3}{32}$, $0.10$ 
  \vspace{0.5in}

\item \hspace{0.5in} $\dfrac{7}{8}$, $\dfrac{7}{16}$, $\dfrac{7}{32}$, $\dfrac{1}{4}$ 
  \vspace{0.5in}

\pagebreak
\end{enumerate}
\end{multicols}

For the remaining problems, express each answer exactly as a fraction in lowest terms, a mixed number in lowest terms, or a whole number. There are twelve inches in one foot, three feet in one yard, and sixteen ounces in one pound.  
\newcommand{\spacing}{\vspace{0.70in}}
\item What fraction of a foot is 8 inches? 
\spacing

\item If a one-foot threaded rod is cut into six equal parts, how long is each part?
\spacing

\item How many inches is one quarter of a foot? 
\spacing

\item How many inches is one ninth of a yard? 
\spacing

\item What fraction of a yard is sixteen inches?  
\spacing

\item How many yards is four feet plus two feet?  
\spacing

\item How many inches is one half of a yard plus one third of a foot?  
\spacing

\item If one box weighs $14$ ounces, and another weighs $1\frac{7}{8}$ pounds, by how many ounces do they differ in weight? 
\spacing

\end{enumerate}

\end{document}
