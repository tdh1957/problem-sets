\documentclass[12pt]{article}
\usepackage{amssymb,amsmath,latexsym,gensymb,mdwlist,tikz,multicol,enumitem}

% Page length commands go here in the preamble
\setlength{\oddsidemargin}{0in} % Left margin of 1 in + 0 in = 1 in
\setlength{\textwidth}{6.5in}   % Right margin of 8.5 in - 1 in - 6.5 in = 1 in
\setlength{\topmargin}{-0.5in}  % Top margin of 2 in -1 in = 1 in
\setlength{\textheight}{9in}  % Lower margin of 11 in - 9 in - 1 in = 1 in

\setlength{\parindent}{0in} %Set the length of the paragraph indentation.

%%%%%%%%%%%%%%%%%%%%%%%%%%%%%%%%%%%%%%%%%%

\begin{document}
\pagestyle{empty} % controls page numbering
\begin{center}
          Problem set - Metric measurements - 2\\[0.5in]
\end{center}
Name: \rule{4in}{0.005in} Date: \rule{1.5in}{0.005in} 
  \vspace{0.25in}

Use dimensional analysis to answer questions 1 through 11. Show all of your work. You may use a calculator and notes. 

There are ten millimeters in one centimeter, and one hundred centimeters in one meter. One kilometer is one thousand meters. 

\newcommand{\spacing}{\vspace{0.70in}}
\begin{enumerate}
\item How many millimeters is 12cm?  
\spacing

\item How many meters is 800cm? 
\spacing

\item How many centimeters is 2.4m? 
\spacing

\item How many meters is 21km?  
\spacing

\item How many millimeters is three meters? 
\spacing

\item How many centimeters is 0.12km? 
\spacing

\pagebreak

		\suspend{enumerate}
One liter is 1000 milliliters. Sixty seconds is one minute. Sixty minutes is one hour. 
		\resume{enumerate}

\item How many milliliters is a half liter? 
\spacing

\item How many seconds is twenty-two minutes? 
\spacing

\item How long does it take a pump with a rated capacity of 0.4 liters per minute to empty a 30l tank? 
\spacing

\item How many kilometers per hour is 2.5 meters per second? 
\spacing

\item How many meters per second is twelve kilometers per hour? 
\spacing

		\suspend{enumerate}
Temperatures in Celsius and Fahrenheit are related by the formulas 
		\begin{equation*}
			C=\dfrac{5}{9}(F-32) \hspace{1in} \text{and} \hspace{1in} F=\dfrac{9}{5}C+32 
		\end{equation*}
		\resume{enumerate}
\spacing

\item If the temperature is 52\degree F, what is the temperature in Celsius? 
\spacing

\item If the temperature is 14\degree C, what is the temperature in Fahrenheit? 
\spacing

\end{enumerate}
\end{document}
