\documentclass[12pt]{article}
\usepackage{amssymb,amsmath,latexsym,gensymb,mdwlist,tikz,multicol,enumitem}

% Page length commands go here in the preamble
\setlength{\oddsidemargin}{0in} % Left margin of 1 in + 0 in = 1 in
\setlength{\textwidth}{6.5in}   % Right margin of 8.5 in - 1 in - 6.5 in = 1 in
\setlength{\topmargin}{-0.5in}  % Top margin of 2 in -1 in = 1 in
\setlength{\textheight}{9in}  % Lower margin of 11 in - 9 in - 1 in = 1 in

\setlength{\parindent}{0in} %Set the length of the paragraph indentation.

%%%%%%%%%%%%%%%%%%%%%%%%%%%%%%%%%%%%%%%%%%

\begin{document}
\pagestyle{empty} % controls page numbering
\begin{center}
          Problem set - percents - 1 \\[0.5in]
\end{center}
Name: \rule{4in}{0.005in} Date: \rule{1in}{0.005in} 
  \vspace{0.25in}

Solve each problem. Show all of your work. You may use a calculator and notes. 

\begin{enumerate}
\item Write the decimal as a percent. 
\begin{multicols}{2}
\begin{enumerate}

\item \hspace{0.50in} $0.75$
  \vspace{0.25in}

\item \hspace{0.50in} $0.8$
  \vspace{0.25in}

\item \hspace{0.50in} $4.00$
  \vspace{0.25in}

\item \hspace{0.50in} $23.4$
  \vspace{0.25in}

\item \hspace{0.50in} $2.008$
  \vspace{0.25in}

\item \hspace{0.50in} $0.075$
  \vspace{0.25in}

\end{enumerate}
\end{multicols}
\item Write the percent as a decimal.
\begin{multicols}{2}
\begin{enumerate}

\item \hspace{0.5in} $5\%$
  \vspace{0.25in}

\item \hspace{0.5in} $252\%$
  \vspace{0.25in}

\item \hspace{0.5in} $91.5\%$
  \vspace{0.25in}

\item \hspace{0.5in} $99\frac{9}{10}\%$
  \vspace{0.25in}
\end{enumerate}
\end{multicols}

\item Write the fraction as a percent rounded to the nearest tenth of a percent.
\begin{multicols}{2}
\begin{enumerate}

\item \hspace{0.5in} $\frac{7}{10}$
  \vspace{0.25in}

\item \hspace{0.5in} $\frac{1}{3}$
  \vspace{0.25in}

\item \hspace{0.5in} $\frac{5}{8}$
  \vspace{0.25in}

\item \hspace{0.5in} $\frac{1}{9}$
  \vspace{0.25in}

\item \hspace{0.5in} $\frac{3}{200}$
  \vspace{0.25in}

\item \hspace{0.5in} $2\frac{1}{2}$
  \vspace{0.25in}

\end{enumerate}
\end{multicols}

\item Write the percent as a proper fraction is lowest terms, or as a mixed number in lowest terms.
\begin{multicols}{2}
\begin{enumerate}

\item \hspace{0.5in} $2\%$
  \vspace{0.25in}

\item \hspace{0.5in} $35\%$
  \vspace{0.25in}

\item \hspace{0.5in} $99\%$
  \vspace{0.25in}

\item \hspace{0.5in} $0.25\%$
  \vspace{0.25in}

\item \hspace{0.5in} $350\%$
  \vspace{0.25in}

\item \hspace{0.5in} $62\frac{1}{2}\%$
  \vspace{0.25in}

\end{enumerate}
\end{multicols}

\item What is twelve percent of sixty kilograms? 
  \vspace{0.50in}

\item Find to the nearest hundredth $9.5\%$ of $73$. 
  \vspace{0.50in}

\item To the nearest cent, what is one hundred twenty-eight percent of \$1,645.00?
  \vspace{0.50in}

\item If $0.2\%$ of 120,000 parts is defective, how many parts are defective? 
  \vspace{0.50in}

\item Twelve kilograms is what percent of one hundred ninety-two kilograms? 
  \vspace{0.50in}

\item What percent of $120$ is $60$?
  \vspace{0.50in}

\item Sixteen dollars is what percent of four dollars?
  \vspace{0.50in}

\item To the nearest tenth of a percent, what percent of $78$ ounces is $63$ ounces?
  \vspace{0.50in}

\item Fifty percent of what number is fifteen? 
  \vspace{0.50in}

\item $110$ is what percent of $100$? 
  \vspace{0.50in}

\item To the nearest hundredth of a percent, $100$ is what percent of $110$? 
  \vspace{0.50in}

\end{enumerate}
\end{document}
