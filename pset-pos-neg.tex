\documentclass[12pt]{article}
\usepackage{amssymb,amsmath,latexsym,gensymb,mdwlist,tikz,multicol,enumitem}
% for tables:
\usepackage{multirow}
\usepackage[normalem]{ulem}
\useunder{\uline}{\ul}{}


% Page length commands go here in the preamble
\setlength{\oddsidemargin}{0in} % Left margin of 1 in + 0 in = 1 in
\setlength{\textwidth}{6.5in}   % Right margin of 8.5 in - 1 in - 6.5 in = 1 in
\setlength{\topmargin}{-0.5in}  % Top margin of 2 in -1 in = 1 in
\setlength{\textheight}{9in}  % Lower margin of 11 in - 9 in - 1 in = 1 in

\setlength{\parindent}{0in} %Set the length of the paragraph indentation.

% add plus-or-minus problems, 38 feet +/= 1 ft

%%%%%%%%%%%%%%%%%%%%%%%%%%%%%%%%%%%%%%%%%%

\begin{document}
\pagestyle{empty} % controls page numbering
\begin{center}
          Problem set - positive and negative numbers \\[0.5in]
\end{center}
Name: \rule{4in}{0.005in} Date: \rule{1.4in}{0.005in} 
  \vspace{0.25in}

\begin{enumerate}

\item If the temperature increases from twenty-five above to forty above zero, by how many degrees does the temperature increase?  
	    \vspace{0.50in}

\item If the temperature increases from five below zero to ten above zero, by how many degrees does the temperature increase?  
	    \vspace{0.50in}

\item If the temperature increases from ten below zero to five below zero, by how many degrees does the temperature increase?  
	    \vspace{0.50in}

\item The temperature one morning was 20 above. In the course of the morning, the temperature rose 15 degrees to its daily high. What was that daily high? 
	    \vspace{0.50in}

\item The temperature one morning was 2 below. In the course of the morning, the temperature rose 6 degrees to its daily high. What was that daily high? 
	    \vspace{0.50in}

\item The high temperature one morning was 12 above. Over the rest of the day, the temperature sank by 20 degrees to its low. What was that low? 
	    \vspace{0.50in}

    \item For a through d, simplify each expression. 
\begin{multicols}{2}
\begin{enumerate}

    \item \hspace{0.250in} $-2+4$
  \vspace{0.25in}

    \item \hspace{0.250in} $-4-7$
  \vspace{0.25in}

    \item \hspace{0.250in} $0-(-7)$
  \vspace{0.25in}

    \item \hspace{0.250in} $10-12$
  \vspace{0.25in}

\end{enumerate}
\end{multicols}

\pagebreak

	\item The clerk spends all day adding cash and invoices to the box, and subtracting cash and invoices from the box. Monday night the box contained five twenties and two invoices for \$50 each, so the total value of the box was zero. 

    \begin{enumerate}
  	\item Positive times positive: On Tuesday, the clerk added two fifties to the box. What was the total value of the box Tuesday night? 
	    \vspace{0.50in}

	\item Negative times positive: On Wednesday, the clerk removed two twenties from the box. What was the total value of the box Wednesday night? 
	    \vspace{0.50in}

	\item Positive times negative: On Thursday, the clerk added four invoices for \$20 each. What was the total value of the box Thursday night? 
	    \vspace{0.50in}

	\item Negative times negative: On Friday, the clerk removed two invoices for \$20 each. What was the total value of the box Friday night? 
	    \vspace{0.50in}
    \end{enumerate}

    \item For a through d, simplify each expression. 
\begin{multicols}{2}
\begin{enumerate}

    \item \hspace{0.250in} $-3*5$
  \vspace{0.25in}

    \item \hspace{0.250in} $-4*(-8)$
  \vspace{0.25in}

    \item \hspace{0.250in} $0*(-5)$
  \vspace{0.25in}

    \item \hspace{0.250in} $(-10)*(-12)$
  \vspace{0.25in}

\end{enumerate}
\end{multicols}

	    \vspace{0.50in}

\item The opposite of a number is what you add to that number so that they sum to zero. The opposite of $4$ is $-4$ because $4+{-4}=0$. Find the opposite of each of these. 
\begin{multicols}{2}
\begin{enumerate}

    \item \hspace{0.250in} $2$
  \vspace{0.25in}

    \item \hspace{0.250in} $7$
  \vspace{0.25in}

    \item \hspace{0.250in} $0$
  \vspace{0.25in}

    \item \hspace{0.250in} $-1$
  \vspace{0.25in}

\end{enumerate}
\end{multicols}

\item A dimension is to be $30$m $\pm$ $1$m. What is the maximum value of the dimension? The minimum? 
\item A dimension must be at most 52 meters and at least 48 meters. Write the dimension as its value plus or minus, as in the problem just above.  
    \end{enumerate}
\end{document} 
