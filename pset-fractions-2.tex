\documentclass[12pt]{article}
\usepackage{amssymb,amsmath,latexsym,gensymb,mdwlist,tikz,multicol,enumitem}

% Page length commands go here in the preamble
\setlength{\oddsidemargin}{0in} % Left margin of 1 in + 0 in = 1 in
\setlength{\textwidth}{6.5in}   % Right margin of 8.5 in - 1 in - 6.5 in = 1 in
\setlength{\topmargin}{-0.5in}  % Top margin of 2 in -1 in = 1 in
\setlength{\textheight}{9in}  % Lower margin of 11 in - 9 in - 1 in = 1 in

\setlength{\parindent}{0in} %Set the length of the paragraph indentation.

%%%%%%%%%%%%%%%%%%%%%%%%%%%%%%%%%%%%%%%%%%

\begin{document}
\pagestyle{empty} % controls page numbering
\begin{center}
          Problem set -- fractions - 2 \\[0.5in]
\end{center}
Name: \rule{4in}{0.005in} Date: \rule{1.5in}{0.005in} 
  \vspace{0.25in}
Solve each problem. Show all of your work. You may use a calculator, textbook, and notes. For problems 1 through 10, put the correct sign, $=$, $<$, or $>$ between each pair of fractions. 

\begin{multicols}{2}
\begin{enumerate}

  \item \hspace{0.25in} $\dfrac{3}{8} \hspace{0.25in} \dfrac{5}{8}$
  \vspace{0.25in}

  \item \hspace{0.25in} $-\dfrac{2}{7} \hspace{0.25in} -\dfrac{1}{7}$
  \vspace{0.25in}

  \item \hspace{0.25in} $\dfrac{1}{8} \hspace{0.25in} \dfrac{1}{10}$
  \vspace{0.25in}

  \item \hspace{0.25in} $\dfrac{3}{8} \hspace{0.25in} \dfrac{6}{16}$
  \vspace{0.25in}

  \item \hspace{0.25in} $-\dfrac{3}{8} \hspace{0.25in} \dfrac{4}{9}$
  \vspace{0.25in}

  \item \hspace{0.25in} $\dfrac{1}{2} \hspace{0.25in} \dfrac{3}{7}$
  \vspace{0.25in}

  \item \hspace{0.25in} $-\dfrac{2}{3} \hspace{0.25in} -\dfrac{3}{4}$
  \vspace{0.25in}

  \item \hspace{0.25in} $\dfrac{5}{8} \hspace{0.25in} \dfrac{9}{16}$
  \vspace{0.25in}

  \item \hspace{0.25in} $-\dfrac{4}{9} \hspace{0.25in} \dfrac{8}{13}$
  \vspace{0.25in}

  \item \hspace{0.25in} $\dfrac{1}{10} \hspace{0.25in} \dfrac{2}{9}$
  \vspace{0.25in}

\end{enumerate}
\end{multicols}

  \vspace{0.5in}

For problems 11, 12, and 13, write the numbers in order from least to greatest.
\begin{multicols}{2}
\begin{enumerate}
\setcounter{enumi}{16}

\item \hspace{0.5in} $\dfrac{1}{2}$, $\dfrac{1}{3}$, $\dfrac{5}{16}$, $\dfrac{1}{8}$ 
  \vspace{0.25in}

\item \hspace{0.5in} $2\dfrac{5}{8}$, $-\dfrac{8}{16}$, $\dfrac{3}{32}$, $\dfrac{1}{10}$ 
  \vspace{0.25in}

\item \hspace{0.5in} $\dfrac{5}{8}$, $\dfrac{7}{16}$, $1\dfrac{5}{32}$, $-\dfrac{4}{7}$ 
  \vspace{0.25in}

\pagebreak
\end{enumerate}
\end{multicols}
For the remaining problems, express each answer exactly as a fraction in lowest terms, a mixed number in lowest terms, or a whole number. There are twelve inches in one foot, three feet in one yard, and sixteen ounces in one pound.  
\begin{enumerate}
\setcounter{enumi}{19}
		\newcommand{\spacing}{\vspace{0.50in}}
\item What fraction of a dozen is 9 eggs? 
\spacing

\item If a one-foot threaded rod is cut into four equal parts, how long is each part?
\spacing

\item How many inches is a quarter foot? 
\spacing

\item How many inches is one third of a foot? 
\spacing

\item How many ounces is half of a pound? 
\spacing

\item How many inches is one half of a foot plus one third of a foot?  
\spacing

\item How many ounces is one half of a pound plus one fourth of a pound?  
\spacing

\item In Acme's parts inventory, a type 700 igniter is $1\frac{3}{4}$ feet long; a type 200 igniter is $1\frac{1}{6}$ feet long. How much longer is a type 700? 
\spacing

\item If one box weighs $12$ ounces, and another weighs $1\frac{1}{8}$ pounds, by how many ounces do they differ in weight? 
\spacing

\end{enumerate}

\end{document}
